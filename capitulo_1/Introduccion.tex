% Es el primer capítulo del trabajo. Debe contener las ideas que dan origen al desarrollo del trabajo y la exposición general del problema. Deberá terminar este capítulo con la enumeración de los objetivos que se pretenden lograr con el trabajo, especificando objetivo general y objetivos específicos.

% Cada sección dentro de un capítulo tendrá una numeración de dos o más números siendo el primero igual al número del capítulo en que éste se encuentra. De preferencia no se considerarán más de 3 números. Por ejemplo si la primera sección del Capítulo 1 es ladescripción del problema, su título será:

\chapter{Introducción}

\section{Descripción de }

% \section{Estructura General}

% El informe deberá adoptar la siguiente estructura de presentación:

% \begin{itemize}
% \item Carátula (con letras negras y logo UFRO oficial).
% \item Comisión Examinadora y evaluación.
% \item Dedicatoria (opcional).
% \item Agradecimientos (opcional).
% \item Resumen.
% \item Índice de Contenido.
% \item Índice de Tablas.
% \item Índice de Figuras.
% \item Capítulo 1. Introducción.
% \item Capítulo 2. Antecedentes Generales.
% \item Capítulo 3. Descripción Actividades Realizadas.
% \item Capítulo 4. Resultados y Discusión.
% \item Capítulo 5. Conclusiones.
% \item Nomenclatura.
% \item Bibliografía.
% \item Anexos (opcional).
% \end{itemize}