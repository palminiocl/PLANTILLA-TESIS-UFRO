% ------------------------------------------------------------------------
% Definiciones de AMS-LaTeX: UFRO Thesis  Ramiro Donoso, Marzo 2017 *
% ------------------------------------------------------------------------
% Paquetes por instalar: srcltx, algorithms, algorithmic, algorithm e idioma español.
\documentclass[12pt,notitlepage]{report}
\usepackage[spanish]{babel}		% Silabeo en español. (corte de palabras!)
\usepackage[utf8]{inputenc}	% Caracteres con acentos.
\usepackage[table,xcdraw]{xcolor}
\usepackage{graphicx}			% Para usar figuras JPG, PNG, etc...
\usepackage[centertags]{amsmath}
\usepackage{amsfonts}
\usepackage{amssymb} %letras matematicas (\mathbb{R})
\usepackage{amsthm}
\usepackage{algorithm}
\usepackage{algorithmic}
\usepackage{newlfont}
\usepackage[inactive]{srcltx}  %SRC Specials for DVI search
\usepackage{booktabs}
\usepackage{nonfloat} % para usar tabcaption y figcaption.... tablas no flotantes.
\usepackage{multirow} % para unir filas
\usepackage{multicol} % para unir columnas
\usepackage{dcolumn}
\usepackage{subfigure} % permite la utilizacion de subfiguras en una misma figura
\usepackage[subfigure]{tocloft}
\usepackage{longtable} % para cortar tablas en mas de una pagina
%\usepackage[justification=centering]{caption} % para caption largos y centrados.
\usepackage[final]{pdfpages}
\usepackage{verbatim}
% Definición de colores personalizados
\definecolor{blue}{RGB}{0, 0, 255}      % Azul para palabras clave
\definecolor{red}{RGB}{255, 0, 0}       % Rojo para cadenas
\definecolor{gray}{RGB}{128, 128, 128}  % Gris para comentarios
\definecolor{green}{RGB}{0, 128, 0}     % Verde para funciones o variables
\definecolor{orange}{RGB}{255, 165, 0}  % Naranja para números
%\usepackage{xcolor}
% ESTILOS PROPIOS
\usepackage{UFRO_Style} % Estilos propios de tablas, figuras y comandos.		
\usepackage[]{UFRO_Tesis}	% Incorpora el uso del paquete UFRO_Tesis
%\usepackage[noLinks]{UFRO_Tesis}	% Activa/Desactiva links en el documento.
%\usepackage{lscape}
%\usepackage{longtable}
\usepackage{pdflscape} % Para rotar una página específica
\usepackage{adjustbox} % Opcional, para ajustar tamaños
\usepackage{array}
%\usepackage{subfiles} % Best loaded last in the preamble
%\usepackage[subpreambles=true]{standalone}
%\usepackage{import}
%\lstloadlanguages{JavaScript}

% \lstdefinelanguage{JavaScript}{
%     keywords={typeof, new, true, false, catch, function, return, null, catch, switch, var, if, in, while, do, else, case, break},
%     keywordstyle=\color{blue}\bfseries,
%     ndkeywords={class, export, boolean, throw, implements, import, this},
%     ndkeywordstyle=\color{darkgray}\bfseries,
%     identifierstyle=\color{black},
%     sensitive=false,
%     comment=[l]{//},
%     morecomment=[s]{/*}{*/},
%     commentstyle=\color{purple}\ttfamily,
%     stringstyle=\color{red}\ttfamily,
%     morestring=[b]',
%     morestring=[b]"
% }

% \lstset{
%     language=JavaScript, % Especifica el lenguaje
%     basicstyle=\ttfamily\small, % Tipo y tamaño de fuente
%     keywordstyle=\color{blue}\bfseries, % Color para palabras clave
%     commentstyle=\color{green!50!black}, % Color para comentarios
%     stringstyle=\color{orange}, % Color para cadenas de texto
%     numbers=left, % Números de línea a la izquierda
%     numberstyle=\tiny, % Tamaño de los números de línea
%     stepnumber=1, % Numeración en cada línea
%     numbersep=5pt, % Espaciado entre el código y los números
%     breaklines=true, % Saltar líneas largas
%     breakatwhitespace=true, % Saltar líneas en espacios
%     frame=single, % Marco alrededor del código
%     captionpos=b, % Posición de la leyenda (debajo del código)
%     tabsize=4, % Tamaño de las tabulaciones
%     showspaces=false, % No mostrar espacios
%     showstringspaces=false, % No mostrar espacios en cadenas de texto
%     morekeywords={function,const,let,var,return,async,await}, % Palabras clave adicionales
%     emph={document,window,console,addEventListener}, % Resaltar funciones comunes
%     emphstyle=\color{purple}\bfseries % Estilo para funciones resaltadas
% }

\usepackage{minted}
% Configuración para minted (requiere -shell-escape al compilar)
\setminted{
    linenos,                % Mostrar números de línea
    breaklines,             % Romper líneas largas
    fontsize=\footnotesize, % Tamaño de fuente
    frame=lines,            % Marco simple
    framesep=2mm            % Espaciado del marco
}
\usemintedstyle{tango}



% Redefinir el formato del contador de códigos para incluir el número de sección
%\renewcommand{\thecodes}{\thechapter.\arabic{codes}}
%\renewcommand{\thecodes}{\arabic{chapter}.\arabic{codes}}

% Redefinir el comando \codecaption
\newcommand{\codecaption}[3][]{%
  \refstepcounter{codes}% Incrementa el contador de códigos
    %\addcontentsline{toc}{codes}{#1}
  
  \addcontentsline{cod}{codes}{\thecodes\tab#2}% Agrega a la lista
  \begin{center}
      Código \thecodes: #2% Texto visible
  \end{center}
}


% Fuzz -------------------------------------------------------------------
% ----------------------------------------------------------------
%\vfuzz2pt % Don't report over-full v-boxes if over-edge is small
%		  % These have to do with the justified spacing of a line.
%\hfuzz2pt % Don't report over-full h-boxes if over-edge is small
% permite identificar mas facilmente las over-full h-boxes, v-boxes
\overfullrule=2mm
\hfuzz=2000pt


%%% INICIO DE DOCUMENTO 
\begin{document}
% ------------------------------------------------------------------------


%%% TIPO DE TESIS %%%
%Elija sólo un tipo de Tesis según su carrera o grado
%\ingEjTesis		%pregrado
%\ingTesis			%pregrado
\ingCivTesis		%pregrado
%\magisterTesis		%postgrado
%\doctorTesis		%postgrado
% ------------------------------------------------------------------------
%%% DATOS DE PORTADA %%%d
\dept{Ciencias de La Computación e Informática}
\faculty{Ingeniería y Ciencias}
\carrera{Ingeniero Civil Informático}
\carreraDos{Ingenier\'ia Civil Inform\'atica}
\title{TITULO}
\titlefoot{title foot}
\author{autor}
\matricula{matricula}
\nivelCursado{XII Nivel}
%\secondAuthor{Nombre Completo del Autor 2} % Sólo si corresponde
%\thirdAuthor{Nombre Completo del Autor 3} % Sólo si corresponde
% ------------------------------------------------------------------------
%%% PAGINA DE FIRMA %%%
\profguia{Dr. Profesor Guia}
\profcorrA{Dr. Profesor Comisión A}
\profcorrB{Dr. Profesor Comisión B}
\directorDept{Dr. Director De departamento}
%\profcorrC{Dr. XXX Rev3} %Sólo en caso de postgrado
% ------------------------------------------------------------------------
%%% PREFACIO %%%
\dedicate{dedicatoria}		% Incluir Dedicatoria		(Opcional)
\ack{capitulo_0/Agradecimientos}           	% Incluir Agradecimientos
\resumenesp{capitulo_0/Resumen}            	% Incluir Resumen
\resumening{capitulo_0/Abstract}			% Incluir Abstract 			(Opcional)

% Genera portada, prefacio e índices.
\makeTitle
% ------------------------------------------------------------------------
%%% CONTENIDO DE LA TESIS %%%
% Inclusión del contenido de la tesis por capítulo.
% Es el primer capítulo del trabajo. Debe contener las ideas que dan origen al desarrollo del trabajo y la exposición general del problema. Deberá terminar este capítulo con la enumeración de los objetivos que se pretenden lograr con el trabajo, especificando objetivo general y objetivos específicos.

% Cada sección dentro de un capítulo tendrá una numeración de dos o más números siendo el primero igual al número del capítulo en que éste se encuentra. De preferencia no se considerarán más de 3 números. Por ejemplo si la primera sección del Capítulo 1 es ladescripción del problema, su título será:

\chapter{Introducción}

\section{Descripción de }

% \section{Estructura General}

% El informe deberá adoptar la siguiente estructura de presentación:

% \begin{itemize}
% \item Carátula (con letras negras y logo UFRO oficial).
% \item Comisión Examinadora y evaluación.
% \item Dedicatoria (opcional).
% \item Agradecimientos (opcional).
% \item Resumen.
% \item Índice de Contenido.
% \item Índice de Tablas.
% \item Índice de Figuras.
% \item Capítulo 1. Introducción.
% \item Capítulo 2. Antecedentes Generales.
% \item Capítulo 3. Descripción Actividades Realizadas.
% \item Capítulo 4. Resultados y Discusión.
% \item Capítulo 5. Conclusiones.
% \item Nomenclatura.
% \item Bibliografía.
% \item Anexos (opcional).
% \end{itemize}
% En este capítulo deben entregarse los fundamentos teóricos que sustentan el trabajo a desarrollar. Debe considerar una revisión bibliográfica de trabajos recientes en el área. El nombre de este capítulo podrá ser Antecedentes Bibliográficos, Fundamentos Teóricos, Fundamentación del Tema u otro relacionado.
\chapter{Antecedentes Generales}
\label{C2}
\section{Inteligencia Artificial Conversacional}

\chapter{Metodología}
\label{C3}
%Su importancia dependerá del trabajo realizado. Podrá tener un nombre diferente como Metodología u otro. Se entregará una descripción de cómo fue realizado el trabajo de tal forma que pueda ser reproducido por cualquier persona. Si se trata de un trabajo de laboratorio, se describirán todos los análisis que dan origen a los datos en la forma en que fueron realizados. Se entregarán además antecedentes y diagramas de los equipos utilizados. En caso de tratarse de un trabajo computacional se indicará que software se utilizó, cuál fue el lenguaje y los procedimientos que tuvieron que realizarse para obtener el producto del trabajo.
\section{Metodolog\'ia de Trabajo en Cascada}
\label{sec:metodologia}

\include{capitulo_4/Actividades_realizadas}
% En este capítulo se presentarán los resultados obtenidos y en forma simultánea se discutirá la validez de éstos. En caso de requerirse se indicarán los procedimientos de tratamiento de la información (o datos) para obtener aquella información relevante al trabajo. Aquí es muy importante el análisis de la información (discusión), la comparación de los resultados obtenidos con aquellos obtenidos por otros grupos de trabajo (resultados publicados) pues esta discusión es la base del planteamiento de las conclusiones del trabajo.
\chapter{Resultados y Discusión}
\label{C5}

%Se listarán las conclusiones que se desprenden de la discusión realizada en el capítulo anterior. Cada conclusión corresponderá a un párrafo, siendo precedido por alguna viñeta.
\chapter{Conclusiones}
\label{C6}
%\include{capitulo_6/Conclusion}

%\abreviaciones{capitulo_0/Abreviaciones}	% Incluir Abreviaciones (Opcional)
% ------------------------------------------------------------------------
%%% INCLUSION DE BIBLIOGRAFIA *.bib %%%
\bibliography{bibliografia/Bibliografia}
% ------------------------------------------------------------------------
%%% INCLUSION DE ANEXOS %%%
\appendix
\chapter{Anexo A}
\label{A}

						% Incluir Anexos (Opcional, pero altamente recomendado)

\end{document}
% ------------------------------------------------------------------------ 